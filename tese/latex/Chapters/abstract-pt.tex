%!TEX root = ../template.tex
%%%%%%%%%%%%%%%%%%%%%%%%%%%%%%%%%%%%%%%%%%%%%%%%%%%%%%%%%%%%%%%%%%%%
%% abstract-pt.tex
%% NOVA thesis document file
%%
%% Abstract in Portuguese
%%%%%%%%%%%%%%%%%%%%%%%%%%%%%%%%%%%%%%%%%%%%%%%%%%%%%%%%%%%%%%%%%%%%

\typeout{NT FILE abstract-pt.tex}%

Garantir que um programa se comporta como esperado é um assunto complexo que tem sido alvo de muitas e diferentes tentativas ao longo da história.
A verificação de software assistida por humanos é o método mais confiável e completo, apesar do seu inerente esforço adicional.

O objetivo deste trabalho é facilitar as provas de programas complexos a partir da relação entre este e um equivalente que seja mais simples de provar.
Se for possível estabelecer que dois programas diferentes são equivalentes, então é altamente provável que reutilizar a especificação do mais simples levará a uma prova mais fácil e rápida do programa mais sofisticado.
A lógica de Hoare relacional abriu caminho para o desenvolvimento de várias técnicas para raciocinar sobre as similaridades de dois programas diferentes.
Neste trabalho, vamos basear a nossa abordagem no conceito de \emph{product programs} para reduzir a verificação relacional a verificação "tradicional".

Além disso, haverá uma descrição do conhecimento prévio necessário para compreender efetivamente a nossa abordagem, tal como a apresentação de outros possíveis métodos para alcançar aquilo a que nos propomos.
Finalmente, descreveremos os detalhes da nossa implementação e demonstraremos as capacidades da nossa ferramenta através de diversos exemplos realistas.

% Palavras-chave do resumo em Português
% \begin{keywords}
% Palavra-chave 1, Palavra-chave 2, Palavra-chave 3, Palavra-chave 4
% \end{keywords}
\keywords{
  Verificação dedutiva \and
  Equivalência de programas \and
  OCaml \and
  Cameleer \and
  Lógica de Hoare relacional \and
  Product programs
}
% to add an extra black line
