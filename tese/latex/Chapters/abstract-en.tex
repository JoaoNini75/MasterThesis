%!TEX root = ../template.tex
%%%%%%%%%%%%%%%%%%%%%%%%%%%%%%%%%%%%%%%%%%%%%%%%%%%%%%%%%%%%%%%%%%%%
%% abstract-en.tex
%% NOVA thesis document file
%%
%% Abstract in English([^%]*)
%%%%%%%%%%%%%%%%%%%%%%%%%%%%%%%%%%%%%%%%%%%%%%%%%%%%%%%%%%%%%%%%%%%%

\typeout{NT FILE abstract-en.tex}

Ensuring that a program functions as intended is a complex issue that has been approached many times and in various ways throughout history.
Human-assisted software verification is the most complete and reliable method, despite its inherent additional effort.

The objective of this work is to ease proofs of complex programs by taking advantage of a relation to an equivalent program that is easier to prove.
If it is possible to establish that two different programs are equivalent, it is also very likely that reusing the simplest specification will lead to a faster and easier proof of the more sophisticated program.
Relational Hoare Logic paved the way to the development of several techniques to reason about the similarities of two different programs.
In this work, we will base our approach on the concept of \emph{product programs} to reduce relational verification into standard verification.

Furthermore, there will be a description of the relevant background in order to effectively comprehend the approach we chose, as well as a presentation of the other possible methods to achieve what we propose.
Finally, we will demonstrate a few good initial results obtained by coding and specifying two different implementations of the same algorithm, using different paradigms.


% Palavras-chave do resumo em Inglês
% \begin{keywords}
% Keyword 1, Keyword 2, Keyword 3, Keyword 4, Keyword 5, Keyword 6, Keyword 7, Keyword 8, Keyword 9
% \end{keywords}
\keywords{
  Deductive verification \and
  Program equivalence \and
  OCaml \and
  Cameleer \and
  Relational Hoare logic \and
  Product programs
}
