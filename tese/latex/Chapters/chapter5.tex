%!TEX root = ../template.tex
%%%%%%%%%%%%%%%%%%%%%%%%%%%%%%%%%%%%%%%%%%%%%%%%%%%%%%%%%%%%%%%%%%%%
%% chapter5.tex
%% NOVA thesis document file
%%
%% Chapter with lots of dummy text
%%%%%%%%%%%%%%%%%%%%%%%%%%%%%%%%%%%%%%%%%%%%%%%%%%%%%%%%%%%%%%%%%%%%

\typeout{NT FILE chapter5.tex}%

%% Estrutura:
%% intro (aprofundar um pouco)
%% background (igual + aprofundar whyrel + dune)
%% state of the art (igual)
%% methodology (explicar a ferramenta)
%% case studies (exemplos mais importantes)
%% conclusion and future work

\chapter{Case Studies}
\label{cha:case_studies}

In this chapter, we showcase several case studies that demonstrate the capabilities of the tool developed in this thesis.
We start by presenting a simple example of what BipLang can do that OCaml cannot.
Then, we focus on 3 case studies that are more complex and, therefore, demonstrate the real potential of bip2ml.



\iffalse
%Multiplies two non-negative numbers using nested loops.
\begin{figure}
\begin{minipage}{\linewidth}
\begin{biplangenv}
  let mult_biplang (n : int, m : int) = begin
    let res = ref 0 in
    let j = ref 0 in

    for i = 0 to n-1 do
      (*@ invariant 0 <= i && i <= n
          invariant !j = 0
          invariant !res = m * i + !j *)

      while !j < m do
        (*@ invariant 0 <= !j && !j <= m
            invariant !res = m * i + !j
            variant   m - !j *)

        res := !res + 1;
        j := !j + 1
      done;

      j := 0
    done;
    
    !res
  end
  (*@ requires n >= 0 && m >= 0
      ensures  result = n * m *)
\end{biplangenv}
\end{minipage}
\caption{Using nested loops to multiply two (non-negative) numbers (BipLang).}
\end{figure}
% TODO: put more space

\begin{figure}
\begin{minipage}{\linewidth}
\begin{gospel}
  let mult_ocaml (n : int) (m : int) =
    let res = ref (0) in
    let j = ref (0) in

    for i = 0 to (n - 1) do
      (*@ invariant 0 <= i && i <= n
          invariant !j = 0
          invariant !res = m * i + !j *)

      while (!j < m) do
        (*@ invariant 0 <= !j && !j <= m
            invariant !res = m * i + !j
            variant   m - !j *)
        res := (!res + 1);
        j := (!j + 1)
      done;

      j := 0
    done;

    !res
  (*@ requires n >= 0 && m >= 0
      ensures  result = n * m *)
\end{gospel}
\end{minipage}
\caption{Using nested loops to multiply two (non-negative) numbers (OCaml).}
\end{figure}

\begin{table}[!h]
\begin{center}
\begin{tabular}{|l|l|l|l|c|}
\hline \multicolumn{2}{|c|}{Proof obligations } & \provername{CVC5 1.0.6} \\ 
\hline
\explanation{VC for mult\_ocaml}  & \explanation{loop invariant init} & \valid{0.02} \\ 
\cline{2-3}
 & \explanation{loop invariant init} & \valid{0.03} \\ 
\cline{2-3}
 & \explanation{loop invariant init} & \valid{0.04} \\ 
\cline{2-3}
 & \explanation{loop invariant init} & \valid{0.06} \\ 
\cline{2-3}
 & \explanation{loop invariant init} & \valid{0.03} \\ 
\cline{2-3}
 & \explanation{loop variant decrease} & \valid{0.04} \\ 
\cline{2-3}
 & \explanation{loop invariant preservation} & \valid{0.03} \\ 
\cline{2-3}
 & \explanation{loop invariant preservation} & \valid{0.02} \\ 
\cline{2-3}
 & \explanation{loop invariant preservation} & \valid{0.05} \\ 
\cline{2-3}
 & \explanation{loop invariant preservation} & \valid{0.03} \\ 
\cline{2-3}
 & \explanation{loop invariant preservation} & \valid{0.05} \\ 
\cline{2-3}
 & \explanation{postcondition} & \valid{0.04} \\ 
\cline{2-3}
 & \explanation{VC for mult\_ocaml} & \valid{0.03} \\ 
\hline 
\end{tabular}
\caption{Multiply two (non-negative) numbers (OCaml) verification results.}
\end{center}
\end{table}


%Demonstrating the use of the match and assert constructions.
\begin{figure}
\begin{minipage}{\linewidth}
\begin{biplangenv}
  type number = 
    | Pos of int
    | Neg of int
    | Zero

  let match_assert_type_biplang (x : int) : int = begin
    let y = 
      if x > 0 then begin Pos (x) end
      else begin
        if x < 0 then begin Neg x end
        else begin Zero end
      end
    in  

    let res = (
      match y with
      | Pos n -> (n * 10)
      | Neg (n) -> (n * 2)
      | Zero -> 0
    ) in
    
    let expr_to_assert = (res == (x * 2)) || (res == (x * 10)) in
    assert (expr_to_assert);
    res
  end
  (*@ ensures x > 0 -> result = x * 10
      ensures x = 0 -> result = 0
      ensures x < 0 -> result = x * 2 *)
\end{biplangenv}
\end{minipage}
\caption{The match, assert and type constructions (BipLang).}
\end{figure}
% TODO: put more space

\begin{figure}
\begin{minipage}{\linewidth}
\begin{gospel}
  type number =
    | Pos of int
    | Neg of int
    | Zero

  let match_assert_type_ocaml (x : int) : int =
    let y = 
      if (x > 0)
      then begin 
        Pos (x)
      end else begin 
        if (x < 0)
        then begin 
          Neg (x)
        end else begin 
          Zero
        end
      end
    in
    let res = (
      match y with
      | Pos (n) -> (n * 10)
      | Neg (n) -> (n * 2)
      | Zero -> 0
    ) in
    let expr_to_assert = ((res = (x * 2)) || (res = (x * 10))) in
    assert (expr_to_assert);
    res
  (*@ ensures x > 0 -> result = x * 10
      ensures x = 0 -> result = 0
      ensures x < 0 -> result = x * 2 *)
\end{gospel}
\end{minipage}
\caption{The match, assert and type constructions (OCaml).}
\end{figure}
% TODO: put more space

\begin{table}[!h]
\begin{center}
\begin{tabular}{|l|l|l|l|c|}
\hline \multicolumn{2}{|c|}{Proof obligations } & \provername{CVC5 1.0.6} \\ 
\hline
\explanation{VC for match\_assert\_type\_ocaml}  & \explanation{assertion} & \valid{0.06} \\ 
\cline{2-3}
 & \explanation{postcondition} & \valid{0.04} \\ 
\cline{2-3}
 & \explanation{postcondition} & \valid{0.04} \\ 
\cline{2-3}
 & \explanation{postcondition} & \valid{0.05} \\ 
\hline
\end{tabular}
\caption{The match, assert and type constructions (OCaml) verification results.}
\end{center}
\end{table}


%Showcasing the presence of nested functions through the use of function applying.
\begin{figure}
\begin{minipage}{\linewidth}
\begin{biplangenv}
  let nested_funs_biplang (arg) : int = begin
    let x = 10 in

    let inner_fun (i : int) : int = begin
      x * i 
    end
    (*@ ensures result = x * i *)
    in

    inner_fun (arg * 5) 
  end
  (*@ ensures result = 50 * arg *)  
\end{biplangenv}
\end{minipage}
\caption{Nested functions and function applying (BipLang).}
\end{figure}
% TODO: put more space

\begin{figure}
\begin{minipage}{\linewidth}
\begin{gospel}
  let nested_funs_ocaml (arg) : int =
    let x = 10 in

    let inner_fun (i : int) : int =
      (x * i)
    (*@ ensures result = x * i *)
    in
      
    inner_fun ((arg * 5))
  (*@ ensures result = 50 * arg *)  
\end{gospel}
\end{minipage}
\caption{Nested functions and function applying (OCaml).}
\end{figure}
% TODO: put more space

\begin{table}[!h]
\begin{center}
\begin{tabular}{|l|l|l|l|c|}
\hline \multicolumn{2}{|c|}{Proof obligations } & \provername{CVC5 1.0.6} \\ 
\hline
\explanation{VC for nested\_funs\_ocaml}  & \explanation{postcondition} & \valid{0.02} \\ 
\cline{2-3}
 & \explanation{postcondition} & \valid{0.03} \\ 
\hline
\end{tabular}
\caption{Nested functions and function applying (OCaml) verification results.}
\end{center}
\end{table}
\fi


\section{Incremeting OCaml}
\label{sec:incrementing}

In this section, we give an example that showcases some of the functionalities and constructions that BipLang adds on top of OCaml.
That is accomplished by the introduction of the \bm{$\langle|\rangle$} (pipe), \bm{$\lfloor$} (left floor) and \bm{$\rfloor$} (right floor) symbols.
We also present our tool outputs so we can compare the differences between writing each of these programs using BipLang versus direct OCaml.

Recall the multiplication \hyperref[fig:mult_source_programs]{unary programs} and their \hyperref[fig:mult_biprogram]{biprogram}.
We also presented before its corresponding \hyperref[fig:translation_ex]{WhyML product program}, the one that would be generated by WhyRel.
Now, we rewrite the biprogram in BipLang and use our transpiler to get the generated OCaml code.
Remember, the rules applied by bip2ml are the same as the ones developed by WhyRel's authors.

As described before, the \emph{mult} biprogram combines two unary programs that multiply two non-negative integer numbers.
On the left side, that is done through two nested while loops, incrementing the result by 1 in each iteration.
On the right side, there is a while loop with an assignment inside, making the result increment by $m$ in each cycle iteration.

The specification of $mult\_biplang$ and $mult\_ocaml$ is the same as $mult\_whyml$, but now written in GOSPEL.
Besides that, since OCaml does not support labels, we had to substitute it with a simple variable that is used later to get the value of the previous iteration.
In practice, it also works, but WhyML's labels may help the readibility in terms of separating code and specification.

The reader may have also noticed that, in $mult\_ocaml$, we declare and initialize the $j\_l$ variable but we never use it.
This can be seen as a limitation of bip2ml: we cannot mix "neutral" variables with "sided" ones.
We also cannot mix left side and right side variables, but that would hardly make sense in the code, although it clearly makes sense in the specification.
In this case, since the condition of the $mult\_biplang$'s while loop has floors around it, those identifiers will be transformed into left side identifiers.
So, even if we declared \emph{let j = ref 0 in}, we would still get \emph{while (!i\_l < n\_l) do}, and that would result in variables' names mismatching in the generated OCaml code.

Regarding the verification of $mult\_ocaml$, CVC5 was able to discharge all VCs automatically in a total of ~410 milliseconds.

\begin{figure}
\begin{minipage}{\linewidth}
\begin{biplangenv}


let mult_biplang (|_ n: int _|, |_ m: int _|) : |_ int _| = begin
  let i = |_ ref 0 _| in
  let res = |_ ref 0 _| in
	
  while |_ !i < n _| do
    (*@ variant   n_l - !i_l
        invariant !i_l = !i_r && !res_l = ! res_r *)

    let previous_res_l = !res_l in
    let j = |_ ref 0 _| in

    (( 
      while (!j < m) do
        (*@ variant   m_l - !j_l
            invariant 0 <= !j_l <= m_l && !res_l = !res_r + !j_l *)
        res := !res + 1;
        j := !j + 1
      done
    )
    <|>
    (res := !res + m));

    assert ((!res_l = previous_res_l + m_l));
    i := |_ !i + 1 _|
  done;

  |_ !res _|
end
(*@ requires n_l = n_r && m_l = m_r && m_l >= 0 
    ensures  match result with (l_res, r_res) -> l_res = r_res *)
\end{biplangenv}
\end{minipage}
\caption{$mult$ biprogram (BipLang).}
\end{figure}

\begin{figure}
\begin{minipage}{\linewidth}
\begin{gospel}


let mult_ocaml (n_l : int) (n_r : int) (m_l : int) (m_r : int) :
  int * int =

  let i_l = ref (0) in
  let i_r = ref (0) in
  let res_l = ref (0) in
  let res_r = ref (0) in

  while (!i_l < n_l) do
    (*@ invariant ((!i_l < n_l)) <-> ((!i_r < n_r))
        variant   n_l - !i_l
        invariant !i_l = !i_r && !res_l = ! res_r *)
    let previous_res_l = !res_l in
    let j_l = ref (0) in
    let j_r = ref (0) in    

    while (!j_l < m_l) do
      (*@ variant   m_l - !j_l
          invariant 0 <= !j_l && !j_l <= m_l && !res_l = !res_r + !j_l *)
      res_l := (!res_l + 1);
      j_l := (!j_l + 1)
    done;
    
    res_r := (!res_r + m_r);
    assert ((!res_l = (previous_res_l + m_l)));
    i_l := (!i_l + 1);
    i_r := (!i_r + 1)
  done;

  (!res_l, !res_r)
(*@ requires n_l = n_r && m_l = m_r && m_l >= 0
    ensures  match result with (l_res, r_res) -> l_res = r_res *)
\end{gospel}
\end{minipage}
\caption{$mult$ product program (OCaml).}
\end{figure}

\begin{table}[!h]
\begin{center}
\begin{tabular}{|l|l|l|l|c|}
\hline \multicolumn{2}{|c|}{Proof obligations } & \provername{CVC5 1.0.6} \\ 
\hline
\explanation{VC for mult\_ocaml}  & \explanation{loop invariant init} & \valid{0.05} \\ 
\cline{2-3}
 & \explanation{loop invariant init} & \valid{0.04} \\ 
\cline{2-3}
 & \explanation{loop invariant init} & \valid{0.02} \\ 
\cline{2-3}
 & \explanation{loop variant decrease} & \valid{0.03} \\ 
\cline{2-3}
 & \explanation{loop invariant preservation} & \valid{0.04} \\ 
\cline{2-3}
 & \explanation{assertion} & \valid{0.06} \\ 
\cline{2-3}
 & \explanation{loop variant decrease} & \valid{0.04} \\ 
\cline{2-3}
 & \explanation{loop invariant preservation} & \valid{0.04} \\ 
\cline{2-3}
 & \explanation{loop invariant preservation} & \valid{0.05} \\ 
\cline{2-3}
 & \explanation{postcondition} & \valid{0.04} \\ 
\hline 
\end{tabular}
\caption{$mult$ product program (OCaml) verification results.}
\end{center}
\end{table}



\iffalse
% TODO why does this figure appear before the text above??
\begin{figure}
\begin{minipage}{\linewidth}
\begin{biplangenv}
  let pipe_biplang (x : int <|> x : int) = begin
    let y = 10 <|> 1 in
    let z = 10 <|> 100 in
    let res = x * y * z <|> x * y * z in
    res <|> res
  end
  (*@ requires x_l = x_r
      ensures  match result with (l_res, r_res) -> 
                (l_res = r_res && l_res = x_l * 100) *)
\end{biplangenv}
\end{minipage}
\caption{Showcasing the use of the pipe symbol (BipLang).}
\end{figure}

\begin{figure}
\begin{minipage}{\linewidth}
\begin{gospel}
  let pipe_ocaml (x_l : int) (x_r : int) =
    let y_l = 10 in
    let y_r = 1 in
    let z_l = 10 in
    let z_r = 100 in
    let res_l = ((x_l * y_l) * z_l) in
    let res_r = ((x_r * y_r) * z_r) in
    (res_l, res_r)
  (*@ requires x_l = x_r
      ensures  match result with (l_res, r_res) -> 
                (l_res = r_res && l_res = x_l * 100) *)
\end{gospel}
\end{minipage}
\caption{Showcasing the use of the pipe symbol (OCaml).}
\end{figure}

\begin{table}[!h]
\begin{center}
\begin{tabular}{|l|l|l|l|c|}
\hline \multicolumn{2}{|c|}{Proof obligations } & \provername{CVC5 1.0.6} \\ 
\hline
\explanation{VC for pipe\_ocaml}  & \explanation{postcondition} & \valid{0.03} \\ 
\hline
\end{tabular}
\caption{Showcasing the use of the pipe symbol (OCaml) verification results.}
\end{center}
\end{table}


\begin{figure}
\begin{minipage}{\linewidth}
\begin{biplangenv}
  let floors_biplang (|_ arg1 : int _|, |_ arg2 : int _|)
    : |_ int _| = begin

    let x = |_ arg1 * 2 _| in
    let y = |_ arg2 _| in

    if |_ y > x _| then begin
      |_ x + y _|
    end else begin
      |_ x - y _|
    end
  end
  (*@ requires arg1_l = arg1_r && arg2_l = arg2_r
      ensures  match result with (l_res, r_res) -> 
                (l_res = r_res && l_res = (
                  if arg2_l > arg1_l * 2
                  then arg1_l * 2 + arg2_l
                  else arg1_l * 2 - arg2_l)) *)
\end{biplangenv}
\end{minipage}
\caption{Demonstrating the use of the floor symbols (BipLang).}
\end{figure}

\begin{figure}
\begin{minipage}{\linewidth}
\begin{gospel}
  let floors_ocaml (arg1_l : int) (arg1_r : int)
    (arg2_l : int) (arg2_r : int) : int * int =
    
    let x_l = (arg1_l * 2) in
    let x_r = (arg1_r * 2) in
    let y_l = arg2_l in
    let y_r = arg2_r in
    assert ( ((y_l > x_l)) = ((y_r > x_r)) );

    if (y_l > x_l)
    then begin 
      ((x_l + y_l), (x_r + y_r))
    end else begin 
      ((x_l - y_l), (x_r - y_r))
    end
  (*@ requires arg1_l = arg1_r && arg2_l = arg2_r
      ensures  match result with (l_res, r_res) -> 
                (l_res = r_res && l_res = (
                  if arg2_l > arg1_l * 2
                  then arg1_l * 2 + arg2_l
                  else arg1_l * 2 - arg2_l)) *)
\end{gospel}
\end{minipage}
\caption{Demonstrating the use of the floor symbols (OCaml).}
\end{figure}

\begin{table}[!h]
\begin{center}
\begin{tabular}{|l|l|l|l|c|}
\hline \multicolumn{2}{|c|}{Proof obligations } & \provername{CVC5 1.0.6} \\ 
\hline
\explanation{VC for floors\_ocaml}  & \explanation{assertion} & \valid{0.04} \\ 
\cline{2-3}
 & \explanation{postcondition} & \valid{0.05} \\ 
\hline
\end{tabular}
\caption{Demonstrating the use of the floor symbols (OCaml) verification results.}
\end{center}
\end{table}
\fi


\FloatBarrier
\section{Real World Cases}
\label{sec:usefulness}

This section illustrates the applicability of our tool through some more complex examples.
As we did in the previous section, we present these programs written in BipLang and their corresponding translations to OCaml.


\FloatBarrier
\subsection{Induction variable strength reduction}
\label{subsec:rwc-ivsr}

The \hyperref[fig:ivsr_biplang]{first} example is a common compiler optimization, the induction variable strength reduction.
Recall the source, optimized and product \hyperref[fig:induction_var_strength_red]{programs}, presented earlier.

\begin{figure}
\begin{minipage}{\linewidth}
\begin{biplangenv}


let induc_var_strength_red_biplang (|_b : int_|,
  |_c : int_|, |_n : int_|) : |_int_| = begin

  let i = |_ ref 0 _| in
  let j = ref 0 <|> ref c in
  let x = |_ ref 0 _| in

  while |_ !i < n _| do
    (*@ variant   n_l - !i_l
        invariant !i_l = !i_r 
        invariant !x_l = !x_r
        invariant !j_r = !i_r * b_r + c_r *)
    j := !i * b + c <|> x := !x + !j;
    x := !x + !j    <|> j := !j + b;
    i := |_ !i + 1 _|
  done;

  |_ !x _|
end
(*@ requires b_l = b_r && c_l = c_r && n_l = n_r
    ensures  match result with (l_res, r_res) -> l_res = r_res *) 
\end{biplangenv}
\end{minipage}
\caption{Induction variable strength reduction (BipLang).}
\label{fig:ivsr_biplang}
\end{figure}

\begin{figure}
\begin{minipage}{\linewidth}
\begin{gospel}


let induc_var_strength_red_ocaml
  (b_l : int) (b_r : int) (c_l : int) (c_r : int)
  (n_l : int) (n_r : int) : int * int =
  
  let i_l = ref (0) in
  let i_r = ref (0) in
  let j_l = ref (0) in
  let j_r = ref (c_r) in
  let x_l = ref (0) in
  let x_r = ref (0) in

  while (!i_l < n_l) do
    (*@ invariant ((!i_l < n_l)) <-> ((!i_r < n_r))
        variant   n_l - !i_l
        invariant !i_l = !i_r 
        invariant !x_l = !x_r
        invariant !j_r = !i_r * b_r + c_r *)
    j_l := ((!i_l * b_l) + c_l);
    x_r := (!x_r + !j_r);
    x_l := (!x_l + !j_l);
    j_r := (!j_r + b_r);
    i_l := (!i_l + 1);
    i_r := (!i_r + 1)
  done;

  (!x_l, !x_r)
(*@ requires b_l = b_r && c_l = c_r && n_l = n_r
    ensures  match result with (l_res, r_res) -> l_res = r_res *)
\end{gospel}
\end{minipage}
\caption{Induction variable strength reduction (OCaml).}
\label{fig:ivsr_ocaml}
\end{figure}

We start by initializing the $i$, $j$ and $x$ variables.
We use the floors for $i$ and $x$, since both sides of the program bind to them the same value (\emph{ref 0}, for both variables).
To initialize $j$, on the other hand, we use the pipe, since one side binds to it \emph{ref 0} and the other \emph{ref c}.

We then have the main part of the program: the floored while loop.
The condition of the cycle indicates that as long as \emph{!i < n} is true for both left and right, it keeps executing.
This gets translated to only the left side condition and the first invariant in the loop of the generated program, automatically introduced by our tool.
The other invariants and the variant had to be added by us.
Then, inside the loop, we have two assignments to different identifiers, which explains the slightly different syntax of the form \emph{id1 := value1 <|> id2 := value2}.
The last instruction of the while is an equal assign to both sides of the program.

Finally, we return the value inside the $x$ reference, for the left and right sides equally, which gets translated to the return of a binary tuple.
The pre-conditions and post-conditions are also provided by the user.
In this case, we are simply saying that agreement on the input results in agreement of the output.

Regarding the proof duration, CVC5 was able to discharge all VCs automatically in 0.3 seconds.

\begin{table}[!h]
\begin{center}
\begin{tabular}{|l|l|l|l|c|}
\hline \multicolumn{2}{|c|}{Proof obligations } & \provername{CVC5 1.0.6} \\ 
\hline
\explanation{VC for induc\_var\_strength\_red}  & \explanation{loop invariant init} & \valid{0.03} \\ 
\cline{2-3}
 & \explanation{loop invariant init} & \valid{0.03} \\ 
\cline{2-3}
 & \explanation{loop invariant init} & \valid{0.03} \\ 
\cline{2-3}
 & \explanation{loop invariant init} & \valid{0.01} \\ 
\cline{2-3}
 & \explanation{loop variant decrease} & \valid{0.05} \\ 
\cline{2-3}
 & \explanation{loop invariant preservation} & \valid{0.03} \\ 
\cline{2-3}
 & \explanation{loop invariant preservation} & \valid{0.03} \\ 
\cline{2-3}
 & \explanation{loop invariant preservation} & \valid{0.03} \\ 
\cline{2-3}
 & \explanation{loop invariant preservation} & \valid{0.02} \\ 
\cline{2-3}
 & \explanation{postcondition} & \valid{0.04} \\ 
\hline 
\end{tabular}
\caption{Induction variable strength reduction (OCaml) verification results.}
\end{center}
\end{table}


\FloatBarrier
\subsection{Loop alignment}
\label{subsec:rwc-la}

TODO: loop alignment
Reference \hyperref[fig:loop_alignment]{???????}
Contrarily to the previous example, this is one is not exactly a proof that the programs output the same for the same input, but rather that, after the execution, a part of the memory of both programs is equal.



\begin{figure}
\begin{minipage}{\linewidth}
\begin{biplangenv}


let loop_alignment_biplang (|_ n : int _|, |_ a : int array _|,
  |_ b : int array _|, |_ d : int array _|) = begin

  let i = |_ ref 1 _| in
  assert (!i_l <= n_l);
  b_l.(!i_l) <- a_l.(!i_l);
  d_l.(!i_l) <- b_l.(!i_l - 1);
  i_l := !i_l + 1;
  d_r.(1) <- b_r.(0);

  while !i_l < n_l do
    (*@ variant   n_l - !i_l
        invariant (!i_l < n_l) <-> (!i_r < n_r - 1)
        invariant !i_r >= 0 && !i_l = !i_r + 1
        invariant b_l.(!i_r) = a_l.(!i_r)
        invariant b_l.(!i_r - 1) = b_r.(!i_r - 1) 
        invariant forall k. 1 <= k < !i_l -> d_l.(k) = d_r.(k) *) 

    |_ b.(!i) <- a.(!i) _|;
    d.(!i) <- b.(!i - 1) <|> d.(!i + 1) <- b.(!i);
    i := |_ !i + 1 _|
  done;

  b_r.(n_r) <- a_r.(n_r)
end
(*@ requires n_l >= 1 && n_l = n_r 
    requires Array.length a_l = n_l + 1 
    requires Array.length b_l = n_l + 1 
    requires Array.length d_l = n_l + 1 

    requires Array.length a_l = Array.length a_r
    requires Array.length b_l = Array.length b_r
    requires Array.length d_l = Array.length d_r

    requires forall k. 0 <= k < n_l -> a_l.(k) = a_r.(k)
    requires b_l.(0) = b_r.(0)
		
    ensures  forall k. 1 <= k < n_l -> d_l.(k) = d_r.(k) *)
\end{biplangenv}
\end{minipage}
\caption{Loop alignment (BipLang).}
\label{fig:la_biplang}
\end{figure}


\begin{figure}
\begin{minipage}{\linewidth}
\begin{gospel}

  
let loop_alignment_ocaml (n_l : int) (n_r : int)
  (a_l : int array) (a_r : int array) (b_l : int array)
  (b_r : int array) (d_l : int array) (d_r : int array) =

  let i_l = ref (1) in
  let i_r = ref (1) in
  assert ((!i_l <= n_l));
  b_l.(!i_l) <- a_l.(!i_l);
  d_l.(!i_l) <- b_l.((!i_l - 1));
  i_l := (!i_l + 1);
  d_r.(1) <- b_r.(0);

  while (!i_l < n_l) do
    (*@ variant   n_l - !i_l
        invariant (!i_l < n_l) <-> (!i_r < n_r - 1)
        invariant !i_r >= 0 && !i_l = !i_r + 1
        invariant b_l.(!i_r) = a_l.(!i_r)
        invariant b_l.(!i_r - 1) = b_r.(!i_r - 1) 
        invariant forall k. 1 <= k < !i_l -> d_l.(k) = d_r.(k) *)    
    b_l.(!i_l) <- a_l.(!i_l);    
    b_r.(!i_r) <- a_r.(!i_r);    
    d_l.(!i_l) <- b_l.((!i_l - 1));    
    d_r.((!i_r + 1)) <- b_r.(!i_r);
    i_l := (!i_l + 1);
    i_r := (!i_r + 1)
  done;

  b_r.(n_r) <- a_r.(n_r)
(*@ requires n_l >= 1 && n_l = n_r 
    requires Array.length a_l = n_l + 1 
    requires Array.length b_l = n_l + 1 
    requires Array.length d_l = n_l + 1 

    requires Array.length a_l = Array.length a_r
    requires Array.length b_l = Array.length b_r
    requires Array.length d_l = Array.length d_r

    requires forall k. 0 <= k < n_l -> a_l.(k) = a_r.(k)
    requires b_l.(0) = b_r.(0)
		
    ensures  forall k. 1 <= k < n_l -> d_l.(k) = d_r.(k) *)
\end{gospel}
\end{minipage}
\caption{Loop alignment (OCaml).}
\label{fig:la_ocaml}
\end{figure}

The proof did not require any human interaction since CVC5 discharged all VCs in approximately 1.82 seconds.

\begin{table}[!h]
\begin{center}
\begin{tabular}{|l|l|l|l|c|}
\hline \multicolumn{2}{|c|}{Proof obligations } & \provername{CVC5 1.0.6} \\ 
\hline
\explanation{VC for loop\_alignment\_biplang}  & \explanation{assertion} & \valid{0.06} \\ 
\cline{2-3}
 & \explanation{index in array bounds} & \valid{0.04} \\ 
\cline{2-3}
 & \explanation{precondition} & \valid{0.05} \\ 
\cline{2-3}
 & \explanation{index in array bounds} & \valid{0.05} \\ 
\cline{2-3}
 & \explanation{precondition} & \valid{0.04} \\ 
\cline{2-3}
 & \explanation{index in array bounds} & \valid{0.05} \\ 
\cline{2-3}
 & \explanation{precondition} & \valid{0.05} \\ 
\cline{2-3}
 & \explanation{loop invariant init} & \valid{0.05} \\ 
\cline{2-3}
 & \explanation{loop invariant init} & \valid{0.02} \\ 
\cline{2-3}
 & \explanation{loop invariant init} & \valid{0.09} \\ 
\cline{2-3}
 & \explanation{loop invariant init} & \valid{0.10} \\ 
\cline{2-3}
 & \explanation{loop invariant init} & \valid{0.04} \\ 
\cline{2-3}
 & \explanation{index in array bounds} & \valid{0.03} \\ 
\cline{2-3}
 & \explanation{precondition} & \valid{0.05} \\ 
\cline{2-3}
 & \explanation{index in array bounds} & \valid{0.06} \\ 
\cline{2-3}
 & \explanation{precondition} & \valid{0.05} \\ 
\cline{2-3}
 & \explanation{index in array bounds} & \valid{0.06} \\ 
\cline{2-3}
 & \explanation{precondition} & \valid{0.06} \\ 
\cline{2-3}
 & \explanation{index in array bounds} & \valid{0.06} \\ 
\cline{2-3}
 & \explanation{precondition} & \valid{0.05} \\ 
\cline{2-3}
 & \explanation{loop variant decrease} & \valid{0.04} \\ 
\cline{2-3}
 & \explanation{loop invariant preservation} & \valid{0.04} \\ 
\cline{2-3}
 & \explanation{loop invariant preservation} & \valid{0.05} \\ 
\cline{2-3}
 & \explanation{loop invariant preservation} & \valid{0.03} \\ 
\cline{2-3}
 & \explanation{loop invariant preservation} & \valid{0.13} \\ 
\cline{2-3}
 & \explanation{loop invariant preservation} & \valid{0.33} \\ 
\cline{2-3}
 & \explanation{index in array bounds} & \valid{0.05} \\ 
\cline{2-3}
 & \explanation{precondition} & \valid{0.05} \\ 
\cline{2-3}
 & \explanation{postcondition} & \valid{0.04} \\ 
\hline 
\end{tabular}
\caption{Loop alignment (OCaml) verification results.}
\end{center}
\end{table}


\FloatBarrier
\subsection{Conditionally alignmed loops}
\label{subsec:rwc-cal}

The third and last real world example is also not exactly a proof of equivalence, but for a different reason.
Instead, it is a proof that the left program's output is always larger than the right side program's output, given the same inputs and \emph{x >= 4}.
REFERENCIAR FIGURE DO CODIGO DOS 
condalignloops DA SECCAO DO WHYREL

\begin{figure}
\begin{minipage}{\linewidth}
\begin{biplangenv}
  TODO: prova ainda nao passa!

  (*@ axiom mult: forall a:int, b:int, c:int, d:int.
     0 < a -> 0 < b -> 0 < c -> 0 < d -> a > b ->
     c > d -> (a * c) > (b * d) *) 
    
  let cond_align_loops_biplang (|_x : int_|, |_n : int_|)
    : |_int_| = begin

    let y = |_ ref x _| in
    let z = ref 24 <|> ref 16 in 
    let w = |_ ref 0 _| in

    while !y > 4 <|> !y > 4 . (!w mod 2 <> 0) <|> (!w mod 2 <> 0) do
      (*@ variant   !y_l + !y_r
          invariant !y_l = !y_r && !y_r >= 4
          invariant !z_l > !z_r && !z_l > 0 && !z_r > 0 *)

      if ((!w mod n == 0) <|> (!w mod n == 0)) then begin
        z := (!z * !y <|> !z * 2);
        y := |_ !y - 1 _|
      end else begin () end;

      w := |_ !w + 1 _|
    done; 

    |_ !z _|
  end
  (*@ requires x_l = x_r && n_l = n_r && n_r > 0 && x_l >= 4
      ensures  match result with (l_res, r_res) -> l_res > r_res *)  
\end{biplangenv}
\end{minipage}
\caption{Conditionally aligned loops (BipLang).}
\end{figure}
% TODO: put more space

\begin{figure}
\begin{minipage}{\linewidth}
\begin{gospel}
  TODO: 
  prova ainda nao passa!
  codigo nao cabe numa pagina!

  (*@ axiom mult: forall a:int, b:int, c:int, d:int.
     0 < a -> 0 < b -> 0 < c -> 0 < d -> a > b ->
     c > d -> (a * c) > (b * d) *)

  let cond_align_loops_ocaml (x_l : int) (x_r : int)
    (n_l : int) (n_r : int) : int * int =

    let y_l = ref (x_l) in
    let y_r = ref (x_r) in
    let z_l = ref (24) in
    let z_r = ref (16) in
    let w_l = ref (0) in
    let w_r = ref (0) in

    while ((!y_l > 4) || (!y_r > 4)) do
      (*@ variant   !y_l + !y_r
          invariant !y_l = !y_r && !y_r >= 4
          invariant !z_l > !z_r && !z_l > 0 && !z_r > 0
          invariant (!y_l > 4 && mod !w_l 2 <> 0) ||
                    (!y_r > 4 && mod !w_r 2 <> 0) ||
                    (not (!y_l > 4) && not (!y_r > 4)) || 
                    (!y_l > 4 && !y_r > 4) *)
      if ((!y_l > 4) && ((!w_l mod 2) <> 0))
      then begin 
        if ((!w_l mod n_l) = 0)
        then begin 
          z_l := (!z_l * !y_l);
          y_l := (!y_l - 1)
        end else begin 
          ()
        end;
        w_l := (!w_l + 1)
      end else begin 
        if ((!y_r > 4) && ((!w_r mod 2) <> 0))
        then begin 
          if ((!w_r mod n_r) = 0)
          then begin 
            z_r := (!z_r * 2);
            y_r := (!y_r - 1)
          end else begin 
            ()
          end;
          w_r := (!w_r + 1)
        end else begin 
          assert ( (((!w_l mod n_l) = 0)) = (((!w_r mod n_r) = 0)) );
          if ((!w_l mod n_l) = 0)
          then begin 
            z_l := (!z_l * !y_l);
            z_r := (!z_r * 2);
            y_l := (!y_l - 1);
            y_r := (!y_r - 1)
          end else begin 
            ()
          end;
          w_l := (!w_l + 1);
          w_r := (!w_r + 1)
        end
      end
    done;

    (!z_l, !z_r)
  (*@ requires x_l = x_r && n_l = n_r && n_r > 0 && x_l >= 4
      ensures  match result with (l_res, r_res) -> l_res > r_res *) 
\end{gospel}
\end{minipage}
\caption{Conditionally aligned loops (OCaml).}
\end{figure}


\iffalse\begin{table}[!h]
\begin{center}

\end{tabular}
\caption{Conditionally aligned loops (OCaml) verification results.}
\end{center}
\end{table}
\fi






\iffalse
\begin{itemize}
    \setlength\itemsep{0bp}

    \item{\emph{Task 1: Collection of case studies}}\hspace{1em}
    In March, we plan to select interesting cases where applying our work would be beneficial. 
    Some examples are compiler optimizations (and optimizations in general), reduction of higher-order functions to first-order and even code versioning, in the sense that refactoring or new features do not modify the behavior of previously correct code.

    \item \emph{Task 2: First-order product programs for OCaml}\hspace{1em} 
    Afterwards, and spanning across a month and a half, we will adapt the original work of product programs (which is based on the \emph{while} language) to the reality of OCaml, in the first-order context.

    \item \emph{Task 3: First-order implementation in Cameleer}\hspace{1em} 
    During two months, we plan on implementing the first-order OCaml approach in the Cameleer tool.

    \item \emph{Task 4: Extension of product programs to higher-order}\hspace{1em} 
    After that, in the month of July, we plan to extend our notion of product programs in OCaml to include some higher-order constructs focused on iteration, such as folds (left and right), map and iter.

    \item \emph{Task 5: Higher-order implementation in Cameleer}\hspace{1em} 
    In the following month, we will attempt to extend the implementation of product programs in Cameleer to include the higher-order functions mentioned in the previous task.

    \item \emph{Task 6: Dissertation Writing}\hspace{1em} 
    Finally, we will dedicate the entire month of September to write the dissertation.

\end{itemize}
  
\begin{figure}[h]
\tikzset{every picture/.style={xscale=0.65,yscale=0.65,transform shape}}
\begin{ganttchart}[ y unit chart = 0.6cm,
                      vgrid,
                      bar top shift=-0.1,
                      bar height=0.6,
                      title height=0.7]{1}{28}
    \gantttitle{March}{4}
    \gantttitle{April}{4}
    \gantttitle{May}{4}
    \gantttitle{June}{4}
    \gantttitle{July}{4}
    \gantttitle{August}{4}
    \gantttitle{September}{4}\\
    \gantttitlelist{1,...,4}{1}
    \gantttitlelist{1,...,4}{1}
    \gantttitlelist{1,...,4}{1}
    \gantttitlelist{1,...,4}{1}
    \gantttitlelist{1,...,4}{1}
    \gantttitlelist{1,...,4}{1}
    \gantttitlelist{1,...,4}{1} \\
  
  \ganttgroup{Preliminary}{1}{4} \\
  \ganttbar{Collection of case studies}{1}{4} \\
  
  \ganttgroup{First-order product programs}{5}{16} \\
  \ganttbar{Definition in OCaml}{5}{10} \\
  \ganttbar{Implementation in Cameleer}{9}{16} \\

  \ganttgroup{Extension to higher-order}{17}{24} \\
  \ganttbar{Definition in OCaml}{17}{20} \\
  \ganttbar{Implementation in Cameleer}{21}{24} \\
  
  \ganttgroup{Dissertation}{25}{28} \\
  \ganttbar{Writing}{25}{28}
  
  \end{ganttchart}
  \caption{Planned Schedule.}
  \label{ganttchart}
\end{figure}
\fi
