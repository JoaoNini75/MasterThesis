%!TEX root = ../template.tex
%%%%%%%%%%%%%%%%%%%%%%%%%%%%%%%%%%%%%%%%%%%%%%%%%%%%%%%%%%%%%%%%%%%%
%% chapter5.tex
%% NOVA thesis document file
%%
%% Chapter with lots of dummy text
%%%%%%%%%%%%%%%%%%%%%%%%%%%%%%%%%%%%%%%%%%%%%%%%%%%%%%%%%%%%%%%%%%%%

\typeout{NT FILE chapter5.tex}%

%% Estrutura:
%% intro (aprofundar um pouco)
%% background (igual + aprofundar whyrel + dune)
%% state of the art (igual)
%% methodology (explicar a ferramenta)
%% case studies (exemplos mais importantes)
%% conclusion and future work

\chapter{Case Studies}
\label{cha:case_studies}


\section{The Foundations}
\label{sec:foundations}

exemplos de programas que poderiam ser escritos apenas em ocaml 


\section{Incremeting OCaml}
\label{sec:incrementing}

exemplos de programas que necessitam das features da biplang (pipe, floor, (mais?))


\section{Demonstrating the Applicability}
\label{sec:applicability}

os exemplos mais dificeis e complexos


\iffalse
\begin{itemize}
    \setlength\itemsep{0bp}

    \item{\emph{Task 1: Collection of case studies}}\hspace{1em}
    In March, we plan to select interesting cases where applying our work would be beneficial. 
    Some examples are compiler optimizations (and optimizations in general), reduction of higher-order functions to first-order and even code versioning, in the sense that refactoring or new features do not modify the behavior of previously correct code.

    \item \emph{Task 2: First-order product programs for OCaml}\hspace{1em} 
    Afterwards, and spanning across a month and a half, we will adapt the original work of product programs (which is based on the \emph{while} language) to the reality of OCaml, in the first-order context.

    \item \emph{Task 3: First-order implementation in Cameleer}\hspace{1em} 
    During two months, we plan on implementing the first-order OCaml approach in the Cameleer tool.

    \item \emph{Task 4: Extension of product programs to higher-order}\hspace{1em} 
    After that, in the month of July, we plan to extend our notion of product programs in OCaml to include some higher-order constructs focused on iteration, such as folds (left and right), map and iter.

    \item \emph{Task 5: Higher-order implementation in Cameleer}\hspace{1em} 
    In the following month, we will attempt to extend the implementation of product programs in Cameleer to include the higher-order functions mentioned in the previous task.

    \item \emph{Task 6: Dissertation Writing}\hspace{1em} 
    Finally, we will dedicate the entire month of September to write the dissertation.

\end{itemize}
  
\begin{figure}[h]
\tikzset{every picture/.style={xscale=0.65,yscale=0.65,transform shape}}
\begin{ganttchart}[ y unit chart = 0.6cm,
                      vgrid,
                      bar top shift=-0.1,
                      bar height=0.6,
                      title height=0.7]{1}{28}
    \gantttitle{March}{4}
    \gantttitle{April}{4}
    \gantttitle{May}{4}
    \gantttitle{June}{4}
    \gantttitle{July}{4}
    \gantttitle{August}{4}
    \gantttitle{September}{4}\\
    \gantttitlelist{1,...,4}{1}
    \gantttitlelist{1,...,4}{1}
    \gantttitlelist{1,...,4}{1}
    \gantttitlelist{1,...,4}{1}
    \gantttitlelist{1,...,4}{1}
    \gantttitlelist{1,...,4}{1}
    \gantttitlelist{1,...,4}{1} \\
  
  \ganttgroup{Preliminary}{1}{4} \\
  \ganttbar{Collection of case studies}{1}{4} \\
  
  \ganttgroup{First-order product programs}{5}{16} \\
  \ganttbar{Definition in OCaml}{5}{10} \\
  \ganttbar{Implementation in Cameleer}{9}{16} \\

  \ganttgroup{Extension to higher-order}{17}{24} \\
  \ganttbar{Definition in OCaml}{17}{20} \\
  \ganttbar{Implementation in Cameleer}{21}{24} \\
  
  \ganttgroup{Dissertation}{25}{28} \\
  \ganttbar{Writing}{25}{28}
  
  \end{ganttchart}
  \caption{Planned Schedule.}
  \label{ganttchart}
\end{figure}
\fi
